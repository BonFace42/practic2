%% -*- coding: utf-8 -*-
\documentclass[14pt,a4paper]{scrartcl} 
\usepackage[utf8]{inputenc}
\usepackage[english,russian]{babel}
\usepackage{indentfirst}
\usepackage{misccorr}
\usepackage{graphicx}
\usepackage{amsmath}
\usepackage{listings}

\begin{document}
\tableofcontents
\newpage
\section{Введение}
Для дискретного преобразования Фурье была использована среда разработки Visual Studio 2022. Для оформления и написания отчёта использовался онлайн-компилятор LaTeX Overleaf


\newpage
\section{Алгоритм решения}
Рассмотрим алгоритм, который позволяет перемножить два полинома
длиной n за время O(n log n), что значительно лучше времени O(n2),
достигаемого тривиальным алгоритмом умножения. Очевидно, что умножение
двух длинных чисел можно свести к умножению полиномов, поэтому два
длинных числа также можно перемножить за время O(n log n).

Следует заметить, что, во-первых, два многочлена следует привести к
одной степени (просто дополнив коэффициенты одного из них нулями). Во-
вторых, в результате произведения двух многочленов степени получается
многочлен степени 2n-1,поэтому, чтобы результат получился корректным,
предварительно нужно удвоить степени каждого многочлена (опять же,
дополнив их нулевыми коэффициентами).
 



\newpage
\section{Программа}
\begin{figure}[h!]
    \centering
    \includegraphics [width=0.9\textwidth]{Код}\\
   
    \label{fig:picResult}
\end{figure}





\newpage \section{Список используемой литературы}

\begin{enumerate}

    \item "Язык программирования C++. Базовый курс" Бьерн Страуструп - Издательство: «Питер», 2006, 1104с.
    \item "Роберт Лафоре. Объектно-ориентированное программирование в С++
   
\end{enumerate}

\end{document}